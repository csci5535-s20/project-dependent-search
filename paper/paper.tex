%% This is an abbreviated template from http://www.sigplan.org/Resources/Author/.

\documentclass[acmsmall,review,authorversion]{acmart}
\acmDOI{}
% \acmJournal{FACMP}
\acmVolume{CSCI 5535}
\acmNumber{Spring 2020}

\usepackage{xargs}                      % Use more than one optional parameter in a new commands
\usepackage[colorinlistoftodos,prependcaption,textsize=tiny]{todonotes}
\newcommandx{\unsure}[2][1=]{\todo[linecolor=red,backgroundcolor=red!25,bordercolor=red,#1]{#2}}
\newcommandx{\change}[2][1=]{\todo[linecolor=blue,backgroundcolor=blue!25,bordercolor=blue,#1]{#2}}
% \newcommandx{\info}[2][1=]{\todo[linecolor=OliveGreen,backgroundcolor=OliveGreen!25,bordercolor=OliveGreen,#1]{#2}}
% \newcommandx{\improvement}[2][1=]{\todo[linecolor=Plum,backgroundcolor=Plum!25,bordercolor=Plum,#1]{#2}}
% \newcommandx{\thiswillnotshow}[2][1=]{\todo[disable,#1]{#2}}

\newcommand{\NC}{(\textbf{Needs Citation})}

\newcommand{\term}[1]{\textsf{#1}}
\newcommand{\type}[1]{\textsf{#1}}

\newcommand{\Carrier}{\term{Carrier}}
\newcommand{\Universe}{\mathcal{U}}
\newcommand{\Set}{\type{Set}}
\newcommand{\Type}{\type{Type}}
\newcommand{\Boolean}{\term{Boolean}}
\newcommand{\true}{\term{true}}
\newcommand{\false}{\term{false}}
\newcommand{\Number}{\term{Number}}
\newcommand{\Interp}[1]{\llbracket #1 \rrbracket}
\newcommand{\?}{\stackrel{?}{\approx}}
\newcommand{\search}{\term{search}}

\newcommand{\data}[2]{\textsf{data } #1 : #2 \textsf{ where}}
\newcommand{\record}[2]{\textsf{record } #1 : #2 \textsf{ where}}

\newcommand{\N}{\mathbb{N}}
\newcommand{\ra}{\rightarrow}

\usepackage[utf8]{inputenc}
\DeclareUnicodeCharacter{27E8}{$\langle$}
\DeclareUnicodeCharacter{27E9}{$\rangle$}
\DeclareUnicodeCharacter{2237}{$::$}
\DeclareUnicodeCharacter{21D2}{$\Rightarrow$}
\begin{document}

%%
%% The "title" command has an optional parameter,
%% allowing the author to define a "short title" to be used in page headers.
\title{Embedded Program Search in Agda}

%%
%% The "author" command and its associa

%% the authors and their affiliations.
%% Of note is the shared affiliation of the first two authors, and the
%% "authornote" and "authornotemark" commands
%% used to denote shared contribution to the research.
\author{Jack Martin}
\email{John.P.Martin@colorado.edu}
\author{Michael Dresser}
\email{Michael.M.Dresser@colorado.edu}
\affiliation{%
  \institution{University of Colorado Boulder}
}


%%
%% The abstract is a short summary of the work to be presented in the
%% article.
\begin{abstract}
The problem of Program Search - looking for a function which satisfies a query -
is one of the more practical tools a developer can use when programming. Search
tools such as Hoogle provide an entrypoint for a language eco-system, but
unfortunately often don't accept query information other than types. Here we
present a search tool for searching over types and examples as well as a general
framework for doing program search over dependent types. We also provide a
type-theoretic interpretation of the program search problem as a special case of
the more general synthesis problem and discuss how this presents a tool for
furthering the solution to both problems.
\end{abstract}

%%
%% This command processes the author and affiliation and title
%% information and builds the first part of the formatted document.
\maketitle

\section{Introduction}

Some of the more powerful tools in use by functional programmers are engines for
searching over code bases given the type signature of the desired function. Such
tools provide the entry-point for practical code reuse and management of large
code bases. Tools such as Hoogle(for Haskell)\cite{mitchell2008hoogle} provide a
best-in-class search functionality to polymorphic functional languages, but the
same tools are not as prevalent for dependently-typed languages.

There are a number of factors that have prevented robust search features from
entering the dependently-typed landscape. The primary setback is the
undecidability of type equality(isomorphism) in the presence of dependent types.
For search over types to be performed there must be some way of saying whether
two types are equal or not. This introduces many of the nuances of equality into
the confusion of trying to build a practical tool.

We attempt to solve this problem by defining an abstraction for
search engines of dependent types that can scale and adapt to appropriate
definitions of type equality. It is reasonable for a user to want some control
over how weak/strong the equalities for search are. Some may want strong
definitional equalities, others may accept the more computationally burdensome(and
semi-decidable) isomorphisms. We give an approach that leaves this control to
the user, and also leaves open the possibility for very powerful forms of
program search.

Another deficiency of current search engines for program source is the lack of
being able to specify complex functional properties over the domain of search.
Input-output examples and algebraic properties offer two great sources of
specification for search, yet they are not used; typically this is because of
the lack of methods for verifying these properties. Fortunately, a dependently
typed language offers a natural way of handling these properties.

Practically, we provide all of the abstractions formalized in
Agda\cite{norell2007towards} presented as a concrete tool for search over terms
with some basic types. Notably, our method is embedded \textit{within} the type
theory such that the search function can be treated just like any other
function. The implementation is capable of being extended to support more
expressive properties which in turn will provide more power to a user to specify
complex queries. In the future, the theories in this paper will be developed
into a search-directed approach to program synthesis.

\section{Overview}

\subsection{Program Search}

\newcommand{\extype}{\ensuremath \N \ra \N \ra \N}

An advantage of the functional programming paradigm is the ability to compose
programs from simpler functions. Large libraries exist containing many of the
simple(and not-so-simple) functions that a programmer would like to avoid
writing. However, the user of a functional language often runs into the
situation where they have an idea of the function they need, but don't know the
name or even where to begin looking. A robust tool for searching over the
available libraries is the obvious solution.

Let's assume we would like to search for a program that operates on two natural
numbers. We have \texttt{plus} in mind, but can't remember the name of the
function nor the module it may be located in. But we do know its behavior and
type. The type of the function is $\extype$. We also can describe its behavior
over a few input-output examples:

$$
\begin{array}{ccc}
  \text{input 1} & \text{input 2} & \text{output}\\
  0 & 0 & 0\\
  2 & 2 & 4\\
  1 & 2 & 3\\
  ... &...&...
\end{array}
$$

The benefit of working in a dependent type theory is that we can encode these
examples directly in our query type. So instead of $\extype$, we have the
\textit{(iterated) dependent sum}:


\begin{align*}
\Sigma[ f \in \extype]\ &(f\ 0\ 0 \equiv 0)\\
                         \times\ &(f\ 2\ 2 \equiv 4)\\
                         \times\ &(f\ 1\ 2 \equiv 3)\\
                         \times\ &...
\end{align*}

This type corresponds to the function $f$ of type $\extype$ along with proofs
that the function satisfies our examples. A term of this type with \texttt{plus}
as the function $f$ could look like:

\begin{align*}
  ( \texttt{plus}\ ,\ \texttt{refl}\ ,\ \texttt{refl}\ ,\ \texttt{refl}\ ,\ ...)
\end{align*}

Where \texttt{refl} is a reflexivity proof that a given example holds. The
search for this term is quite simple. First, search for a term matching the type
of the query, then check if the examples are satisfied by a reflexivity proof.
Finding a term of the query type is fairly straightforward, but to find a proof
that each example is held requires a bit more. Notably, the type of the output
has a decidable (judgemental) equality. This is a property that would typically
be proven in the same module as the types definition. Because of this, we can't
make any assumptions about a given type possessing such a decidable equality. We
should make the search procedure general enough that it can search for that
decision procedure.

This example is given to point out that the search methods we aim to define must
be able to interact, to some degree, with the modules they search over. Simply
looking for a matching type is not enough. For these reasons our search methods
are embedded \textit{within} the type system as opposed to externally in the
form of a tool or executable.


\section{Preliminaries}

\subsection{Dependently Typed Programming in Agda}

This subsection briefly introduces the necessary ideas from dependently-typed
programming necessary to understand the rest of the paper. Our paper will use
a notation that roughly corresponds to that of the theorem prover Agda. We use
datatype declaration of the form:

$$
\begin{array}{l}
\data{T}{\tau_1 \ra \tau_2 \ra ... \ra \Set}\\
\quad \begin{array}{rl}
  \textsf{C}_1 & : T\ e_{1,1}\ e_{1,2}\ ...\\
  \textsf{C}_1 & : \mathbb{N} \ra T\ e_{2,1}\ e_{2,2}\ ...\\
        ... &\\
  \textsf{C}_n & : T\ e_{n,1}\ e_{n,2}\ ...
\end{array}
\end{array}
$$

Where $T$ is the name of the type(family) being declared. The types $\tau_i$
represent the indices for a type family. $\Set$ is the type of types.
$\textsf{C}_i$ are the constructors for the newly declared type and the
right-hand side of the constructor definition is the type of the constructor
along with the specific type in the family(specified with terms $e_{i,j}$ from
the indice-types) to which it belongs.

We specify iterated sums as record types in the style of Agda. These can be
thought of as representing mathematical structures that are composed of several
elements of different(potentially dependent) types. An example is the semigroup
record: 

$$
\begin{array}{l}
 \record{\textsf{IsSemigroup } (A : \Set)}{\Set}\\
  \quad
  \begin{array}{rl}
    \_\cdot\_ &: A \ra A \ra A\\
    \textsf{assoc} &: \textsf{Associative } (\_\cdot\_)
  \end{array}
\end{array}
$$

This is parameterized over a type $A$ and defines a type containing a binary
operation $\_\cdot\_$ along with a proof that $\_\cdot\_$ is associative. Note
that the term $\textsf{assoc}$ is dependent on the term $\_\cdot\_$. The
underscores in the semigroup operation denote positions for arguments to the
operation. Agda is unique from most languages in that it supports this style of
mixfix operators. As an example, applying $\_\cdot\_$ to arguments $a$ and $b$
would be $a \cdot b$.

\subsection{Universes} One of the advantages of dependent type theories is their
ability to formalize their own type system and meta-theories. These techniques
typically use a structure called a \textit{Universe}, which is defined as a type
along with an interpretation function which takes an element of that type and
computes some type in the type system. Formally:
 
$$
\begin{array}{l}
\record{\Universe}{\Set}\\
\quad  \begin{array}{rl}
         \Carrier &: \Set \\
         \Interp{\_} &: \Carrier \rightarrow \Set
       \end{array}
\end{array}
$$

Here, \textsf{Carrier} is a type containing codes. These codes correspond to
types in Agda's type system and $\Interp{\_}$ is the interpretation function,
which takes an element of \textsf{Carrier} and computes its corresponding
Agda-type. An example of such a universe containing booleans and natural numbers
is given here:

$$
\begin{array}{cc}
\begin{array}{l}
\data{\Type}{\Set}\\
\quad \begin{array}{rl}
        \textsf{Number} :& \Type\\
        \textsf{Boolean} :& \Type\\
      \end{array}
\end{array}
&
\begin{array}{l}
  \Interp{\_}_{\Type} : \Type \rightarrow \Set\\
  \Interp{\textsf{Number}}_{\Type}     = \mathbb{N}\\
  \Interp{\textsf{Boolean}}_{\Type}    = \mathbb{B}\\
\end{array}
\end{array}
$$

We will spend most of the rest of the paper defining a suitable $\Carrier$ type
for simulating the type system enough for type-based search.


\section{Embedding Search}

\subsection{Simulating First Class Modules}

We will need a coarse simulation of Agda's module structure to store entries for
search. This will be accomplished by collecting elements of a $\Carrier$ type
along with the terms they encode into a homogeneous list. The first element will
be the coded type, and the second element(which depends on the first), will be
the term of the type that is interpreted from the code:

$$
\begin{array}{l}
\textsf{Module } : \Set\\
\textsf{Module} = \textsf{List } (\Sigma[ \tau \in \Carrier ]\ \Interp{ \tau })
\end{array}
$$

We can construct an example of a module using our example universe from the
previous section:

$$
\begin{array}{rl}
  \textsf{example} =& (\Boolean\ ,\ \true)\\
                  ::& (\Number\ ,\ 5 )\\
                  ::& (\Boolean\ ,\ \false )\\
                  ::&[]
\end{array}
$$

This new data structure will allow for us to work with terms along with their
types as first class entities as long as we can encode them into this module
structure.


\subsection{Search over First Class Modules}


To support basic type search over these modules we will first have to discuss
what it means for a query type to ``match'' a type in a collection. This clearly
introduces the need for a notion of equality over types, which complicates
matters quickly. Which form of equality of types is appropriate for use in
program search? Fortunately, there has been some research done on this problem
and the current accepted equality is isomorphism of
types\cite{dicosmo2012isomorphisms}. This will equate curried function types to
their uncurried counterparts. For example(i.e. $(A \times B) \rightarrow C
\approx A \rightarrow B \rightarrow C$).

A problem with using isomorphism over dependent types is its
undecidability\cite{HoTT}. The field is currently studying certain decidable
subsets of the problem thus we opt to maximize flexibility by parameterizing our
methods over a provided decidable type equality. The problem of type-directed
program search then just becomes the act of running the decision procedure for
this equality over the coded-types in our module. Thus, a searchable universe
can be defined:

$$
\begin{array}{l}
  \record{\Universe_S}{\Set}\\
  \quad \begin{array}{ll}
          U &: \Universe\\
         \_\approx\_ &: (a\ b\ : \Carrier\ U) \rightarrow \Set\\
         \approx-\textsf{isEquiv} &: \textsf{IsEquivalence } (\Carrier\ U)\ \_\approx\_\\
         \_\?\_ &: \forall (a\ b : \Carrier\ U) \rightarrow \textsf{Dec } (a \approx b)
        \end{array}
\end{array}
$$

Where \textsf{IsEquivalence} is a proposition that $\_\approx\_$ is an
equivalence over the \textsf{Carrier} type and $\Universe$ is the universe
record-type defined above. Now we are ready to define an abstract search
procedure for modules of a decidable universe. The type of the search function
is:

$$
\begin{array}{l}
  \search : (m : \textsf{Module})\rightarrow (c : \Carrier) \rightarrow \textsf{List }\Interp{c}
\end{array}
$$

The function takes in a module of the universe $m$, and a coded-type $c$. The
output type of the function is a list of inhabitants of the interpretation of
$c$. In the remainder of the paper we will construct a concrete Universe with
which we will implement a small embedded search procedure.

\section{Searching over a Subset of Agda} 

To implement the formalism given in the previous section and create a practical
search procedure, we begin by defining the universe of typing-codes
corresponding to the \textsf{Carrier} field of the record type for universes.
This definition is given in Figure \ref{fig:codes}. $U$ is the code for terms of
type $\Set$. $\#$ is the reference type which is used to refer to a type that is
defined elsewhere in the module, an example of its use will be shown below.
$\_\Rightarrow\_$ is the code for function types. The natural number that
indexes \textsf{TypeCode} is used for book-keeping purposes to ensure we don't
ever over-index a module or create any circular references. It is a technical
detail that we will not elaborate on.

Modules need to be redefined(See Figure \ref{fig:mods}) from the abstraction in
the previous section because they allow for codes to reference other
definitions. We will define Module similarly to the type synonym we used above,
except that it will be mutually defined with the interpretation function. Note
that for the $\_::\_$ case we provide a module first in the constructor so that
the entry may depend on it. The entries in the module will be similar to that in
the example module in Section 4. In the same way the dependent pair
did in the previous Module definition.

The interpretation function is given in Figure \ref{fig:interp}. As explained
before, this function is responsible for mapping elements of $\textsf{TypeCode}$
to their respective types in $\Set$. The cases are fairly straightforward except
for references. The reference constructor $\#$ contains a proof that the index
does not go over the length of the list. This is constrained by the indices
mentioned before. The proof includes with it the position of the type we wish to
reference and then the interpretation function looks it up via a recursive call. 

To make all of this a bit more concrete, Figure \ref{fig:ex} is an example of a
module encoded into this typing universe. Here each entry is preceded by its
type in the coded type system. Note that the references properly index the types
of their functions or terms.

\begin{figure}[t]
  \centering
$$  
  \begin{array}{l}
    \data{\textsf{TypeCode}}{\mathbb{N} \rightarrow \Set}\\
    \quad
    \begin{array}{ll}
      U &: \forall \{ n \} \rightarrow \textsf{TypeCode } n\\
      \# &: \forall \{ n\ m \} \rightarrow n < m \rightarrow \textsf{TypeCode } m\\
      \_\Rightarrow\_ &: \forall \{ n \}\ (t_1\ t_2 : \textsf{TypeCode } n)
                        \rightarrow \textsf{TypeCode } n
    \end{array}

  \end{array}
  $$
  \caption{The Type Codes for our Universe supporting Search}
  \label{fig:codes}
\end{figure}


\begin{figure}[b]
  $$
  \begin{array}{l}
    \data{\textsf{Module}}{\mathbb{N} \rightarrow \Set}\\
    \quad \begin{array}{ll}
            []        &: \textsf{Module } 0\\
            \_::\_,\_ &: \forall \{ n \}\\
                      &\rightarrow  (\Gamma : \textsf{Module } n)\\
                      &\rightarrow  (t : \textsf{TypeCode } n)\\
                      &\rightarrow  \Interp{ t }_\Gamma \\
                      &\rightarrow \textsf{Module (suc }  n)
          \end{array}
  \end{array}
$$
\caption{Modules Re-Defined}
\label{fig:mods}
\end{figure} 


  \begin{figure}
    \centering
  $$
  \begin{array}{l}
    \Interp{\_}\_ : \forall \{ n \} \rightarrow \textsf{TypeCode } n \rightarrow
    \textsf{Module } n \rightarrow \Set\\
    \Interp{ U }_\Gamma = \Set\\
    \Interp{ t_1 \Rightarrow t_2 }_\Gamma = \Interp{t_1}_\Gamma \rightarrow \Interp{t_2}_\Gamma\\
    \Interp{ \#\ \textsf{suc-leq-suc ( zero-leq-n)}}_{(\Gamma :: U ,
    \textsf{term})} = \textsf{term}\\
    \Interp{ \#\ \textsf{suc-leq-suc ( zero-leq-n)}}_{(\Gamma :: t ,
    \textsf{term})} = \langle\textsf{Error Case - Impossible under assumptions}\rangle\\
    \Interp{ \#\ \textsf{suc-leq-suc ( suc-leq-suc } n )}_{(\Gamma :: t ,
    \textsf{term})} = \Interp{ \#\ \textsf{suc-leq-suc } n }_\Gamma
  \end{array}
  $$
  \caption{The Interpretation Function}
  \label{fig:interp}
\end{figure}

\begin{figure}
  \centering
\begin{verbatim}
  ex : Module _
  ex = [] ∷ U             , Nat
          ∷ ⟨ 1 ⟩         , 5
          ∷ ⟨ 1 ⟩         , 2
          ∷ U             , Bool
          ∷ ⟨ 4 ⟩         , true
          ∷ ⟨ 4 ⟩         , false
          ∷ ⟨ 4 ⟩ ⇒ ⟨ 4 ⟩ , not
\end{verbatim}
  \caption{A Sample Module.
  The $\langle\_\rangle$ used in this example corresponds to a shorthand for
  generating the safety proof automatically instead of writing it out long-hand
  with $\#$. These can be read as references where the number inside the
  brackets is the location of the type to which it refers.}
  \label{fig:ex}
\end{figure}

% \newpage

\section{Empirical Evaluation - Ease of Use}

This formalism thus far is lacking ergonomics. The tool is rather impractical
because the module encoding needs to be written by hand in a coded language that
relies heavily on numerical indexing. We alleviate this problem with a set of
programs which both generate the coded modules and the coded queries in
\textsf{TypeCode}. These two programs hook into the Agda Compiler to parse the
target library files and generate an auxiliary file containing the coded modules
and queries. The queries will need to be provided as codes in \textsf{TypeCode}.
The user provides the syntax valid for Agda and the tool generates the element
of \textsf{TypeCode} corresponding to the type provided by converting its base
types to the proper references and function arrows to the coded-arrows.

Along with these programs we have created an integration with emacs to run the
indexing program whenever emacs saves an Agda file. The query executable can be
called using a shortcut or from the list of available emacs commands. We have
found that the inclusion of these features made the search procedure practical
for the end-user.

\section{Related Work}

\paragraph{Program Search} Program Search, or the search of functions in some
library, has led to such tools as Hoogle, and is the main category that this
work could be included under. In the past, none of these tools use any
functional properties or examples to help guide the search. Ours is the first
tool to provide this functionality in a general framework. It is also the first,
to our knowledge, to formalize the task of type-directed search using decidable
properties of types.

\paragraph{Typing Universes} There has been much work on universes for generic
programming. Benke et al. create a collection of category-inspired universes
specifically with the aim of formalizing past work on generic programming in the
vein of Generic Haskell \cite{benke2003universes}. This work differs from ours
in its aim, it mainly attempts to automate the writing of functions such as
generic map or fold which have a common behavior no matter the structure of the
type. We attempt to be generic over the types themselves but with a more
concrete aim. This makes this approach to abstract for our use. Another line of
work aims at formalizing type theory in type theory using quotient inductive
types\cite{altenkirch2016type}. This work is closer to our own in that they
encode their universe mutually in a similar way to our encoding. However this
work aims at verifying type theory within itself and makes heavy use of Homotopy
Type Theory\cite{HoTT}. The formalism itself is not quite adequate for our
purposes as it lacks an mechanism for interpreting its type encodings.

\section{Future Work}

\subsection{Ergonomic Improvements}

Our toolchain for generating a module from Agda files will always be apt for
improvement. Ideally, there will be a full integration with the Agda compiler's
interactive mode. This will speed up the generation program as well as eliminate
the need for separate executables.

Supporting the generation of first class modules from multiple files is also a
near term improvement that is technically not too difficult. The name spaces
will need to be qualified and import conflicts resolved. This will open up the
practical use of our tool for searching over a large repository of code such as
the Agda Standard Library.\cite{bove2009brief} 


\subsection{Universe for Fully Dependent Types}

One of the items lacking in this project was a successful encoding of a universe
for fully dependent types. There was a prototype produced that had some
deficiencies preventing it from having a sound interpretation function. This
will be the first follow-up work to be conducted. Completing this universe will
let the search procedure handle example-based searches such as was show in the
overview section. This need not stop at examples however. Any decidable property
over terms/functions can be used in the search process. Thus this formalism will
be much stronger than any program search tools currently available for
functional languages.

\subsection{Program Search as Synthesis}

A key realization motivating this line of research is the relation of program
search to the more general problem of program synthesis. Program Synthesis is
the task of generating a program which satisfies some specification. Types
provide one such specification. If the synthesis task is then reduced to
producing some term of a given query type, the synthesis procedure must produce
an abstract syntax tree that typechecks to the query type. If we restrict the
solution to have an AST of size 1 it essentially forces the synthesis problem to
become program search.

This is not a particularly profound idea. However, when considering program
search as foundational operation in synthesis procedures it becomes more
motivating. This was the reasoning behind formalizing search within Agda as
opposed to as a separate tool. With a function we can compose search with other
functions to build a complex framework for synthesis over dependent types. This
is our main line of future work in this area.


\section{Conclusion}

We have developed a formalism for and implemented the foundations of a search
tool over Agda codebases. This tool allows Agda programmers to search modules
for terms that match a user-specified type. It can be readily expanded to handle
Agda's full type system. Following this expansion, a series of work will be
opened up that will both enable Agda programmers and eventually lead to
development of dependently-typed synthesis.

\newpage
%%
%% The acknowledgments section is defined using the "acks" environment
%% (and NOT an unnumbered section). This ensures the proper
%% identification of the section in the article metadata, and the
%% consistent spelling of the heading.

%%
%% The next two lines define the bibliography style to be used, and
%% the bibliography file.
\bibliographystyle{ACM-Reference-Format}
\bibliography{paper}
\end{document}
